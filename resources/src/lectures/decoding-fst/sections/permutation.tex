
\frame{
	\frametitle{Reordering}
	
	\only<1>{
	Our model of translational equivalences assumes monotonicity 
	\begin{itemize}
		\item a word replacement model 	
		\item {\color{red}operates in {\bf monotone} left-to-right order}		
		\item with no insertions or deletions
		\item constrained to known word-to-word bilingual mappings \\
	(rule set)
	\end{itemize}
	}
	\only<2->{
	Not anymore!
	\begin{itemize}
		\item a word replacement model	
		\item {\color{blue}operates in {\bf arbitrary} order}
		\item with no insertions or deletions
		\item constrained to known word-to-word bilingual mappings \\
	(rule set)
	\end{itemize}
	}
}

\frame{
	\frametitle{Translating arbitrary permutations}
		
\begin{columns}
\begin{column}{0.5\linewidth}
\only<1->{\ftext{nosso amigo comum} \\}
\only<1->{
\scalebox{0.5}{
\begin{tikzpicture}[->,>=stealth',shorten >=1pt,auto,node distance=2.8cm,semithick]
\tikzstyle{every state}=[draw=black,text=black]
\tikzstyle{every path}=[draw=blue,text=blue]	

\node[initial,state,style={initial text=}] (A) {$0$};
\node[state] (B) [right of=A] {$1$};
\node[state] (C) [right of=B] {$2$};
\node[state,accepting] (D) [right of=C] {$3$};
	\path
		(A) edge [bend right] node {\etext{our}} (B)
			edge [bend left] node {\etext{ours}} (B)
		(B) edge [bend right] node {\etext{mate}} (C)
			edge [bend left] node {\etext{friend}} (C)
		(C) edge node {\etext{common}} (D)
			edge [bend right] node {\etext{usual}} (D)
			edge [bend left] node {\etext{ordinary}} (D)
			edge [bend right=60] node [below] {mutual} (D);
\end{tikzpicture} 
}
}

\only<3->{\ftext{nosso comum amigo} \\}
\only<3->{
\scalebox{0.5}{
\begin{tikzpicture}[->,>=stealth',shorten >=1pt,auto,node distance=2.8cm,semithick]
\tikzstyle{every state}=[draw=black,text=black]
\tikzstyle{every path}=[draw=blue,text=blue]	

\node[initial,state,style={initial text=}] (A) {$0$};
\node[state] (B) [right of=A] {$1$};
\node[state] (C) [right of=B] {$2$};
\node[state,accepting] (D) [right of=C] {$3$};
	\path
		(A) edge [bend right] node {\etext{our}} (B)
			edge [bend left] node {\etext{ours}} (B)
		(B) edge node {\etext{common}} (C)
			edge [bend right] node {\etext{usual}} (C)
			edge [bend left] node {\etext{ordinary}} (C)
			edge [bend right=60] node [below] {mutual} (C)
		(C) edge [bend right] node {\etext{mate}} (D)
			edge [bend left] node {\etext{friend}} (D);
	
\end{tikzpicture} 
}
}

\only<5->{\ftext{amigo comum nosso} \\}
\only<5->{
\scalebox{0.5}{
\begin{tikzpicture}[->,>=stealth',shorten >=1pt,auto,node distance=2.8cm,semithick]
\tikzstyle{every state}=[draw=black,text=black]
\tikzstyle{every path}=[draw=blue,text=blue]	

\node[initial,state,style={initial text=}] (A) {$0$};
\node[state] (B) [right of=A] {$1$};
\node[state] (C) [right of=B] {$2$};
\node[state,accepting] (D) [right of=C] {$3$};
	\path
		(A) edge [bend right] node {\etext{mate}} (B)
			edge [bend left] node {\etext{friend}} (B)
		(B) edge node {\etext{common}} (C)
			edge [bend right] node {\etext{usual}} (C)
			edge [bend left] node {\etext{ordinary}} (C)
			edge [bend right=60] node [below] {mutual} (C)
		(C) edge [bend right] node {\etext{our}} (D)
			edge [bend left] node {\etext{ours}} (D);
\end{tikzpicture} 
}
}
\end{column}
\begin{column}{0.5\linewidth}
\only<2->{\ftext{amigo nosso comum} \\}
\only<2->{
\scalebox{0.5}{
\begin{tikzpicture}[->,>=stealth',shorten >=1pt,auto,node distance=2.8cm,semithick]
\tikzstyle{every state}=[draw=black,text=black]
\tikzstyle{every path}=[draw=blue,text=blue]	

\node[initial,state,style={initial text=}] (A) {$0$};
\node[state] (B) [right of=A] {$1$};
\node[state] (C) [right of=B] {$2$};
\node[state,accepting] (D) [right of=C] {$3$};
	\path
		(A) edge [bend right] node {\etext{mate}} (B)
			edge [bend left] node {\etext{friend}} (B)
		(B) edge [bend right] node {\etext{our}} (C)
			edge [bend left] node {\etext{ours}} (C)
		(C) edge node {\etext{common}} (D)
			edge [bend right] node {\etext{usual}} (D)
			edge [bend left] node {\etext{ordinary}} (D)
			edge [bend right=60] node [below] {mutual} (D);			
\end{tikzpicture} 
}
}
\only<4->{\ftext{comum nosso amigo} \\}
\only<4->{
\scalebox{0.5}{
\begin{tikzpicture}[->,>=stealth',shorten >=1pt,auto,node distance=2.8cm,semithick]
\tikzstyle{every state}=[draw=black,text=black]
\tikzstyle{every path}=[draw=blue,text=blue]	

\node[initial,state,style={initial text=}] (A) {$0$};
\node[state] (B) [right of=A] {$1$};
\node[state] (C) [right of=B] {$2$};
\node[state,accepting] (D) [right of=C] {$3$};
	\path
		(A) edge node {\etext{common}} (B)
			edge [bend right] node {\etext{usual}} (B)
			edge [bend left] node {\etext{ordinary}} (B)
			edge [bend right=60] node [below] {mutual} (B)
		(B) edge [bend right] node {\etext{our}} (C)
			edge [bend left] node {\etext{ours}} (C)				
		(C) edge [bend right] node {\etext{mate}} (D)
			edge [bend left] node {\etext{friend}} (D);

\end{tikzpicture} 
}
}

\only<6->{\ftext{comum amigo nosso} \\}	
\only<6->{
\scalebox{0.5}{
\begin{tikzpicture}[->,>=stealth',shorten >=1pt,auto,node distance=2.8cm,semithick]
\tikzstyle{every state}=[draw=black,text=black]
\tikzstyle{every path}=[draw=blue,text=blue]	

\node[initial,state,style={initial text=}] (A) {$0$};
\node[state] (B) [right of=A] {$1$};
\node[state] (C) [right of=B] {$2$};
\node[state,accepting] (D) [right of=C] {$3$};
	\path
		(A) edge node {\etext{common}} (B)
			edge [bend right] node {\etext{usual}} (B)
			edge [bend left] node {\etext{ordinary}} (B)
			edge [bend right=60] node [below] {mutual} (B)
		(B) edge [bend right] node {\etext{mate}} (C)
			edge [bend left] node {\etext{friend}} (C)
		(C) edge [bend right] node {\etext{our}} (D)
			edge [bend left] node {\etext{ours}} (D);
\end{tikzpicture} 
}
}

\end{column}
\end{columns}
	
\only<7>{$3! = 3 \times 2 \times 1 = 6$ permutations}
\only<8>{each has $2 \times 2 \times 4 = 16$ translations \\}
\only<9>{amounting to $6 \times 16 = 96$ solutions}
\only<10>{$I!$ permutations $\times$ $t^I$ translations}


}


\frame{
	\frametitle{Packing permutations}
	
	%\scalebox{0.7}{%
\begin{tikzpicture}[->,>=stealth',shorten >=1pt,auto,node distance=2.8cm,semithick]
\tikzstyle{every state}=[draw=black,text=black]
\node[initial,state,style={initial text=}] (q000) {\only<1-4>{$0$}\only<5->{$000$}};
\only<1-5>{\node[state] (q1) [right of=q000] {$1$};}
\only<1-7>{\node[state] (q2) [right of=q1] {$2$};}
\only<1-7>{\node[state,accepting] (q3) [right of=q2] {$3$};}
\only<5->{\node[state] (q100) [above right of=q000] {$100$};}
\only<6->{
\node[state] (q010) [right of=q000] {$010$};
\node[state] (q001) [below right of=q000] {$001$};
}
\only<7->{\node[state] (q110) [above of=q2] {$110$};}
\only<8->{
\node[state] (q101) [right of=q010] {$101$};
\node[state] (q011) [below of=q2] {$011$};
\node[state,accepting] (q111) [right of=q101] {$111$};
}

\only<1>{\path[color=gray,text=gray] 
(q000) edge node {nosso} (q1);}
\only<1-2>{\path[color=gray,text=gray] 
(q1) edge node {amigo} (q2);}
\only<1-3>{\path[color=gray,text=gray] 
(q2) edge node {comum} (q3);}

\only<2-4>{\path[color=red,text=red] 
(q000) edge [bend left] node {nosso} (q1);}
\only<2-5>{\path[color=red,text=red] 
(q000) edge node {amigo} (q1);}
\only<2-5>{\path[color=red,text=red] 
(q000) edge [bend right] node {comum} (q1);}


\only<3-6>{
\path[color=red,text=red] 
	(q1) edge [bend left] node {nosso} (q2);}
\only<3-5>{
\path[color=red,text=red] 
	(q1) edge node {amigo} (q2);}
\only<3-7>{
\path[color=red,text=red] 
	(q1) edge [bend right ]node {comum} (q2);}
		
\only<4-7>{
\path[color=red,text=red] 
	(q2) edge [bend left] node {nosso} (q3);}
\only<4-7>{
\path[color=red,text=red] 
	(q2) edge node {amigo} (q3);}
\only<4-6>{
\path[color=red,text=red] 
	(q2) edge [bend right ]node {comum} (q3);}
		
\only<5->{\path[color=red,text=red] 
(q000) edge [bend left] node {nosso} (q100);}

\only<5-6>{
\path[color=red,text=red] 
	(q100) edge [bend left=90] node {amigo} (q2);
}
\only<5-7>{
\path[color=red,text=red] 
	(q100) edge [bend left=30] node {comum} (q2);
}
\only<6->{\path[color=red,text=red] 
(q000) edge [bend right] node {\ftext{comum}} (q001);}
\only<6->{\path[color=red,text=red] 
(q000) edge node {amigo} (q010);}


\only<6-7>{
\path[color=red,text=red] 
(q001) edge [bend right=30] node {\ftext{nosso}} (q2);
}
\only<6-7>{
\path[color=red,text=red] 
(q001) edge [bend right=90] node [below] {\ftext{amigo}} (q2);
}
\only<7->{
\path[color=red,text=red] 
(q100) edge [bend left] node {\ftext{amigo}} (q110)
(q010) edge [bend right] node [below] {\ftext{nosso}} (q110)
(q110) edge [bend left] node {\ftext{comum}} (q3);
}
\only<8->{
\path[color=red,text=red] 
(q100) edge [bend left] node [above] {\ftext{comum}} (q101)
(q010) edge node {\ftext{comum}} (q011)
(q001) edge [bend right] node {\ftext{nosso}} (q101)
(q001) edge [bend right] node [below] {\ftext{amigo}} (q011)
(q101) edge node {\ftext{amigo}} (q111)
(q011) edge [bend right] node [below] {\ftext{nosso}} (q111);
}
\end{tikzpicture}
%}

	
}

\frame{
	\frametitle{Packing permutations}

	\begin{textblock*}{80mm}(0.5\textwidth,0.5\textheight)
	
\scalebox{0.6}{
\begin{tikzpicture}[->,>=stealth',shorten >=1pt,auto,node distance=2.8cm,semithick]
\tikzstyle{every state}=[draw=black,text=black]
\tikzstyle{every path}=[draw=red,text=red]

\node[initial,state,style={initial text=}] (q000) {$000$};
\node[state] (q100) [above right of=q000] {$100$};
\node[state] (q010) [right of=q000] {$010$};
\node[state] (q001) [below right of=q000] {$001$};
\node[state] (q110) [above right of=q010] {$110$};
\node[state] (q101) [right of=q010] {$101$};
\node[state] (q011) [below right of=q010] {$011$};
\node[state,accepting] (q111) [right of=q101] {$111$};

\path 
(q000) edge node {\ftext{nosso}} (q100)
(q000) edge node {\ftext{amigo}} (q010)
(q000) edge node {\ftext{comum}} (q001)
%
(q100) edge [bend left] node {\ftext{amigo}} (q110)
(q100) edge [bend left] node [below] {\ftext{comum}} (q101)
%
(q110) edge [bend left] node {\ftext{comum}} (q111)
%
(q010) edge [bend right] node [below] {\ftext{nosso}} (q110)
(q010) edge node {\ftext{comum}} (q011)
%
(q001) edge [bend right] node {\ftext{nosso}} (q101)
(q001) edge [bend right] node [below] {\ftext{amigo}} (q011)
%
(q101) edge node {\ftext{amigo}} (q111)
%
(q011) edge [bend right] node [below] {\ftext{nosso}} (q111);

\end{tikzpicture} 
}

	\end{textblock*}
	

%	Each state represents a set in the powerset of $\{1,2,\ldots, I\}$\\
	Powerset (\emph{all subsets}) of $\{1, 2, \ldots, I\}$
	\begin{itemize}
		\item $2^I$ subsets \\
		think of a vector of $I$ bits ;)
	\end{itemize}
	Intersection with translation rules
	\begin{itemize}
		\item $O(2^I)$ states
		\item $O(t \times I\times 2^I)$ transitions
		\item $O(t^I \times I!)$ paths
	\end{itemize}	
	
	
}

\frame{
	\frametitle{Deductive logic}
	
	
	\begin{columns}
	\begin{column}{0.4\linewidth}
	
\newcommand{\powersetcond}{
\begin{array}{l }
\textcolor<4>{ForestGreen}{1 \leq i \leq I} \\
\textcolor<4>{Red}{c_i = \bzero}
\end{array}
}



\begin{math} %\small
 \left. 
  \begin{array}{l}
    \alert<2>{\textsc{Item}} ~~~~ {\itembrack{\textcolor<2>{blue}{\{0,1\}^I}}} \\
    %\\
    \alert<5>{\textsc{Goal}} ~~~~ {\textcolor<5>{blue}{\itembrack{1^I}}} \\
    %\\  
    \alert<3>{\textsc{Axiom}} \\ 
	\drule{}{\itembrack{\textcolor<3>{blue}{0^I}}}{} \\
    %\\  
    \alert<4>{\textsc{Expand}} \\ 
	\drule{\textcolor<4>{blue}{\itembrack{C}}}{\textcolor<4>{Fuchsia}{\itembrack{\alpha_i(C)}}}{\powersetcond} \\ 
	~~ \text{\small where $\alpha_i(C)$ is a copy of $C$ with $c_i = \bone$}

  \end{array} 
\right.
\end{math}


	\end{column}
	\begin{column}{0.6\linewidth}
		\only<1>{
		Template
		\begin{itemize}
			\item items \ra states
			\item deduction rules \ra transitions
		\end{itemize}
		}
		\only<2>{
		\begin{itemize}
			\item a subset of $\{1, \ldots, I\}$\\
			encoded as a \textcolor{blue}{bit vector of length $I$}
			
		\end{itemize}
		}
		\only<3>{
		\begin{itemize}
			\item we start with an empty sentence \\
			e.g. $I=3 \ra 0^3 = \textcolor{blue}{000}$ 
		\end{itemize}
		}
		\only<4>{
		\begin{itemize}
			\item and continue one word at a time\\
			e.g. $\textcolor{blue}{\itembrack{\textcolor{Red}{0}00}} \textcolor{ForestGreen}{(i=1)} \ra \textcolor{Fuchsia}{\itembrack{100}}$ \\
		\end{itemize}
		}
		\only<5>{
		\begin{itemize}
			\item until we have a complete sentence\\
			e.g. \textcolor<5>{blue}{$\itembrack{111}$}
		\end{itemize}
		}

	\end{column}
	\end{columns}
}


\frame{
	\frametitle{Instantiated deductive logic program}
	
	\begin{textblock*}{80mm}(0.7\textwidth,0.15\textheight)
	\scalebox{0.8}{
\newcommand{\powersetbcond}{
\begin{array}{l }
1 \leq i \leq I \\
\textcolor<6,10,14>{BurntOrange}{
	c_i = \bzero
}
\end{array}
}



\begin{math} %\small
 \left. 
  \begin{array}{l}
    \textsc{Item} ~~~~ {\itembrack{\{0,1\}^I}} \\
    %\\
    \alert<19>{\textsc{Goal}} ~~~~ {\textcolor<19>{red}{\itembrack{1^I}}} \\
    %\\  
    \alert<2>{\textsc{Axiom}} \\ 
	\drule{}{\itembrack{\textcolor<2>{blue}{0^I}}}{} \\
    %\\  
    \alert<3-18>{\textsc{Expand}} \\ 
	\drule{%
		\textcolor<18>{Magenta}{
		\textcolor<17>{Orange}{
		\textcolor<16>{Goldenrod}{
		\textcolor<12-14>{ForestGreen}{
		\textcolor<9-11>{ProcessBlue}{
		\textcolor<6-8>{Fuchsia}{
		\textcolor<3-5>{blue}{
			\itembrack{C}
		}}}}}}}
	}%
	{%
		\textcolor<16-18>{red}{
		\textcolor<11,13>{Magenta}{
		\textcolor<9>{Goldenrod}{
		\textcolor<8,12>{Orange}{
		\textcolor<7>{Goldenrod}{
		\textcolor<5>{ForestGreen}{
		\textcolor<4>{ProcessBlue}{
		\textcolor<3>{Fuchsia}{
			\itembrack{\alpha_i(C)}
		}}}}}}}}
	}%
	{\powersetbcond} \\ 
	%~~ \text{\small where $\alpha_i(C)$ is a copy of $C$ with $c_i = \bone$}

  \end{array} 
\right.
\end{math}

}
	\end{textblock*}
	
	\begin{small}
		Source: \ftext{nosso}$_1$ \ftext{amigo}$_2$ \ftext{comum}$_3$ \\
	\only<2->{
	\texttt{Axiom} \\
	~ {\color{blue}\itembrack{000}} \\
	}
	\only<3->{
	\texttt{Expand} \\
	~ ${\color{blue}\itembrack{000} }(i=1) \ra {\color{Fuchsia}\itembrack{100}}$  \\
	}
	\only<4->{
	~ ${\color{blue}\itembrack{000}} (i=2) \ra {\color{ProcessBlue}\itembrack{010}}$  \\
	}
	\only<5->{
	~ ${\color{blue}\itembrack{000}} (i=3) \ra {\color{ForestGreen}\itembrack{001}}$  \\
	}
	\only<6-14>{
	~ ${\color{Fuchsia}\itembrack{100}} (i=1)$ \xmark  \\
	}
	\only<7->{
	~ ${\color{Fuchsia}\itembrack{100}} (i=2) \ra {\color{Goldenrod}\itembrack{110}}$  \\
	}
	\only<8->{
	~ ${\color{Fuchsia}\itembrack{100}} (i=3) \ra {\color{Orange}\itembrack{101}}$  \\
	}
	\only<9->{
	~ ${\color{ProcessBlue}\itembrack{010}} (i=1) \ra {\color{Goldenrod}\itembrack{110}} $  \\
	}
	\only<10-14>{
	~ ${\color{ProcessBlue}\itembrack{010}} (i=2)$ \xmark \\
	}
	\only<11->{
	~ ${\color{ProcessBlue}\itembrack{010}} (i=3) \ra {\color{Magenta}\itembrack{011}}$  \\
	}
	\only<12->{
	~ ${\color{ForestGreen}\itembrack{001}} (i=1) \ra {\color{Orange}\itembrack{101}} $  \\
	}
	\only<13->{
	~ ${\color{ForestGreen}\itembrack{001}} (i=2) \ra {\color{Magenta}\itembrack{011}} $  \\
	}
	\only<14>{
	~ ${\color{ForestGreen}\itembrack{001}} (i=3)$ \xmark \\
	}
	%\only<15->{
	%~ ${\color{Goldenrod}\itembrack{110}} (i=1..2)$ \xmark \\
	%}
	\only<16->{
	~ ${\color{Goldenrod}\itembrack{110}} (i=3) \ra {\color{red}\itembrack{111}}$ \\
	}
	%\only<17->{
	%~ ${\color{Orange}\itembrack{101}} (i=1,3)$ \xmark \\
	%}
	\only<17->{
	~ ${\color{Orange}\itembrack{101}} (i=2) \ra {\color{red}\itembrack{111}}$  \\
	}
	\only<18->{
	~ ${\color{Magenta}\itembrack{011}} (i=1) \ra {\color{red}\itembrack{111}}$  \\
	}
	%\only<20->{
	%~ ${\color{Magenta}\itembrack{011}} (i=2,3) $  \xmark \\
	%}
	\only<19>{
	\texttt{Goal} \\
	~ {\color{red}\itembrack{111}} \\
	}
	\end{small}
	
	\begin{textblock*}{80mm}(0.45\textwidth,0.5\textheight)
		\scalebox{0.6}{\begin{tikzpicture}[->,>=stealth',shorten >=1pt,auto,node distance=2.8cm,semithick]
\tikzstyle{every state}=[draw=black]
\tikzstyle{every path}=[draw=red,text=red]

\only<2->{\node[initial,state,style={initial text=},text=blue] (q000) {$000$};}
\only<3->{\node[state] (q100) [above right of=q000,text=Fuchsia] {$100$};}
\only<4->{\node[state] (q010) [right of=q000,text=ProcessBlue] {$010$};}
\only<5->{\node[state] (q001) [below right of=q000,text=ForestGreen] {$001$};}
\only<7->{\node[state] (q110) [above right of=q010,text=Goldenrod] {$110$};}
\only<8->{\node[state] (q101) [right of=q010,text=Orange] {$101$};}
\only<11->{\node[state] (q011) [below right of=q010,text=Magenta] {$011$};}
\only<16->{\node[state,accepting] (q111) [right of=q101,text=red] {$111$};}

\only<3->{\path (q000) edge node {\ftext{nosso}} (q100);}
\only<4->{\path (q000) edge node {\ftext{amigo}} (q010);}
\only<5->{\path (q000) edge node {\ftext{comum}} (q001);}
%
\only<7->{\path (q100) edge [bend left] node {\ftext{amigo}} (q110);}
\only<8->{\path (q100) edge [bend left] node [below] {\ftext{comum}} (q101);}
%
%
\only<9->{\path (q010) edge [bend right] node [below] {\ftext{nosso}} (q110);}
\only<11->{\path (q010) edge node {\ftext{comum}} (q011);}
%
\only<12->{\path (q001) edge [bend right] node {\ftext{nosso}} (q101);}
\only<13->{\path (q001) edge [bend right] node [below] {\ftext{amigo}} (q011);}
%
\only<16->{\path (q110) edge [bend left] node {\ftext{comum}} (q111);}
%
\only<17->{\path (q101) edge node {\ftext{amigo}} (q111);}
%
\only<18->{\path (q011) edge [bend right] node [below] {\ftext{nosso}} (q111);}

\end{tikzpicture} }
	\end{textblock*}

}

\frame{
	\frametitle{Word replacement with unconstrained reordering}
	
	
	
	\only<2>{
	\begin{textblock*}{100mm}(0.1\textwidth,0.2\textheight)
	\scalebox{1.2}{
	
\scalebox{0.6}{
\begin{tikzpicture}[->,>=stealth',shorten >=1pt,auto,node distance=2.8cm,semithick]
\tikzstyle{every state}=[draw=black,text=black]
\tikzstyle{every path}=[draw=red,text=red]

\node[initial,state,style={initial text=}] (q000) {$000$};
\node[state] (q100) [above right of=q000] {$100$};
\node[state] (q010) [right of=q000] {$010$};
\node[state] (q001) [below right of=q000] {$001$};
\node[state] (q110) [above right of=q010] {$110$};
\node[state] (q101) [right of=q010] {$101$};
\node[state] (q011) [below right of=q010] {$011$};
\node[state,accepting] (q111) [right of=q101] {$111$};

\path 
(q000) edge node {\ftext{nosso}} (q100)
(q000) edge node {\ftext{amigo}} (q010)
(q000) edge node {\ftext{comum}} (q001)
%
(q100) edge [bend left] node {\ftext{amigo}} (q110)
(q100) edge [bend left] node [below] {\ftext{comum}} (q101)
%
(q110) edge [bend left] node {\ftext{comum}} (q111)
%
(q010) edge [bend right] node [below] {\ftext{nosso}} (q110)
(q010) edge node {\ftext{comum}} (q011)
%
(q001) edge [bend right] node {\ftext{nosso}} (q101)
(q001) edge [bend right] node [below] {\ftext{amigo}} (q011)
%
(q101) edge node {\ftext{amigo}} (q111)
%
(q011) edge [bend right] node [below] {\ftext{nosso}} (q111);

\end{tikzpicture} 
}

	}
	\end{textblock*}
	}
	
	\only<3>{
	\begin{textblock*}{100mm}(0.1\textwidth,0.1\textheight)
	\scalebox{0.9}{
	
\scalebox{0.6}{
\begin{tikzpicture}[->,>=stealth',shorten >=1pt,auto,node distance=2.8cm,semithick]
\tikzstyle{every state}=[draw=black,text=black]
\tikzstyle{every path}=[draw=blue,text=blue]

\node[initial,state,style={initial text=}] (q000) at (0,-3) {$000$};
\node[state] (q100) at (5,0) {$100$};
\node[state] (q010) at (5,-3) {$010$};
\node[state] (q001) at (5,-6) {$001$};
\node[state] (q110) at (10,2) {$110$};
\node[state] (q101) at (12,-1) {$101$};
\node[state] (q011) at (10,-6) {$011$};
\node[state,accepting] (q111) at (15,-3) {$111$};

\path 
(q000) edge [bend left=15] node {\etext{our}} (q100)
(q000) edge [bend left=30] node {\etext{ours}} (q100)

(q000) edge [bend left=15] node {\etext{mate}} (q010)
(q000) edge [bend right=15] node {\etext{friend}} (q010)

(q000) edge  node {\etext{usual}} (q001)
(q000) edge [bend right=15] node {\etext{mutual}} (q001)
(q000) edge [bend right=30] node {\etext{ordinary}} (q001)
(q000) edge [bend right=45] node {\etext{common}} (q001)

%
(q100) edge [bend left=15] node {\etext{friend}} (q110)
(q100) edge [bend left=30] node {\etext{mate}} (q110)

(q100) edge node [below] {\etext{usual}} (q101)
(q100) edge [bend right=15] node [below] {\etext{mutual}} (q101)
(q100) edge [bend right=30] node [below] {\etext{ordinary}} (q101)
(q100) edge [bend right=45] node [below] {\etext{common}} (q101)

%
(q110) edge [bend left=75] node {\etext{usual}} (q111)
(q110) edge [bend left=60] node {\etext{mutual}} (q111)
(q110) edge [bend left=45] node {\etext{ordinary}} (q111)
(q110) edge [bend left=30]node {\etext{common}} (q111)

%
(q010) edge [bend left=15] node [above] {\etext{our}} (q110)
(q010) edge [bend right=15] node [above] {\etext{ours}} (q110)

(q010) edge [bend left=15] node {\etext{usual}} (q011)
(q010) edge [bend left=30] node {\etext{mutual}} (q011)
(q010) edge [bend left=45] node {\etext{ordinary}} (q011)
(q010) edge [bend left=60] node {\etext{common}} (q011)
%
(q001) edge [bend right=90,looseness=1.5] node [above] {\etext{our}} (q101)
(q001) edge [bend right=100,looseness=1.5] node [below] {\etext{ours}} (q101)
(q001) edge node [below] {\etext{friend}} (q011)
(q001) edge [bend left=15] node [above] {\etext{mate}} (q011)
%
(q101) edge [bend left=15] node {\etext{friend}} (q111)
(q101) edge [bend right] node [below] {\etext{mate}} (q111)
%
(q011) edge [bend left=15] node [above] {\etext{our}} (q111)
(q011) edge node [below] {\etext{ours}} (q111);

\end{tikzpicture} 
}

	}
	\end{textblock*}
	}

	\begin{textblock*}{100mm}(0.1\textwidth,0.8\textheight)	
	\small
	Source: \ftext{nosso amigo comum} 
	\begin{enumerate}
		\item<2-> arbitrary permutations $O(I2^I)$
		\item<3-> intersection with the rule set  $O(tI2^I)$
	\end{enumerate}
	\end{textblock*}
}


\frame{
	\frametitle{Problem!}
	
	Before we even discuss a parameterisation of the model we already ran into a tractability issue!
	\pause
	\begin{itemize}
		\item NP-complete \precite{Knight, 1999}
		\item generalised TSP
	\end{itemize}
	\pause
	Direction
	\begin{itemize}
		\item is it sensible to consider the space of {\bf all permutations}?
	\end{itemize}
	\pause
	Solution
	\pause
	\begin{itemize}
		\item constrain reordering {\bf :D}
		\pause
		\item {\bf 0.o} but how?
	\end{itemize}
}

\frame{
	\frametitle{Ad-hoc distortion limit}
	
	Several flavours of distortion limit \precite{Lopez, 2009}
	\pause
	\begin{itemize}
		\item limit reordering as a function of the number of skipped words
	\end{itemize}
	\pause
	Moses allows reordering within a window of length $d$
	\begin{itemize}
		\item starting from the leftmost uncovered word
	\end{itemize}
	
}

\frame{
	\frametitle{WL$d$ (intuition)}
	Suppose a sentence with $I=6$ words and $d=3$\\
	
	~
	
	\textcolor{gray}{
	\begin{tabular}{|c | c | c | c | c | c |}
	\hline
	\textcolor{black}{1} & \textcolor{black}{2} & \textcolor{black}{3} & \textcolor{black}{4} & \textcolor{black}{5} & \textcolor{black}{6} \\ \hline
	\textcolor{red}{0} & \textcolor{blue}{\only<2-3>{0}\only<4-5>{1}} & \textcolor{blue}{\only<2,5>{0}\only<3-4>{1}}  & 0 & 0 & 0 \\
	1 & \textcolor{red}{0} & \textcolor{blue}{\only<6-7>{0}\only<8-9>{1}}  & \textcolor{blue}{\only<6,9>{0}\only<7-8>{1}} & 0 & 0 \\
	1 & 1 & \textcolor{red}{0} & \textcolor{blue}{\only<10-11>{0}\only<12-13>{1}} & \textcolor{blue}{\only<10,13>{0}\only<11-12>{1}} & 0 \\
	1 & 1 & 1 & \textcolor{red}{0} & \textcolor{blue}{\only<14-15>{0}\only<16-17>{1}} & \textcolor{blue}{\only<14,17>{0}\only<15-16>{1}} \\
	1 & 1 & 1 & 1 & \textcolor{red}{0} & \textcolor{blue}{\only<18>{0}\only<19>{1}} \\
	1 & 1 & 1 & 1 & 1 & \textcolor{red}{0} \\	
	1 & 1 & 1 & 1 & 1 & 1 \\		
	\hline
	\end{tabular}
	}
	
	~
	
	
	\only<20>{
	Largely reduced set of permutations
	\begin{itemize}
		\item $(I-1)2^{d-1}$ configurations
	\end{itemize}
	}
}


\frame{
	\frametitle{WL$d$ (example)}
	
	Suppose $d=2$ and $I=3$ 
	\only<10>{(e.g. \ftext{nosso amigo comum})} \\

~


\scalebox{0.8}{
\begin{tikzpicture}[->,>=stealth',shorten >=1pt,auto,node distance=2.8cm,semithick]
\tikzstyle{every state}=[draw=black,text=black]
\tikzstyle{every path}=[draw=red,text=red]

\only<2->{\node[initial,state,style={initial text=}] (q10) {$\{1,2\}$};}
\only<3->{\node[state] (q20) [above right of=q10] {$\{2,3\}$};}
\only<4->{\node[state] (q11) [below right of=q10] {$\{1,\not2\}$};}
\only<5->{\node[state] (q3) [right of=q11] {$\{3 \}$};}
\only<7->{\node[state] (q21) [right of=q20] {$\{2,\not3\}$};}
\only<8->{\node[state,accepting] (q4) [below right of=q21] {$\varnothing$};}



\only<3-9>{\path (q10) edge node {\ftext{1}} (q20);}
\only<4-9>{\path (q10) edge node {\ftext{2}} (q11);}
\only<5-9>{\path (q11) edge node {\ftext{1}} (q3);}
\only<6-9>{\path (q20) edge node {\ftext{2}} (q3);}
\only<7-9>{\path (q20) edge node {\ftext{3}} (q21);}
\only<8-9>{\path (q21) edge node {\ftext{2}} (q4);}
\only<9>{\path (q3) edge node {\ftext{3}} (q4);}

\only<10>{
\path (q10) edge node {\ftext{nosso}} (q20);
\path (q10) edge node {\ftext{amigo}} (q11);
\path (q20) edge node {\ftext{comum}} (q21);
\path (q20) edge node {\ftext{amigo}} (q3);
\path (q11) edge node {\ftext{nosso}} (q3);
\path (q3) edge node {\ftext{comum}} (q4);
\path (q21) edge node {\ftext{amigo}} (q4);
}



\end{tikzpicture} 
}


	
}


\frame{
	\frametitle{WL$d$ (logic)}
	
	
	\begin{columns}
	\begin{column}{0.4\linewidth}
	
\newcommand{\wldadjcond}{
\begin{array}{l }
i = l\\
\end{array}
}

\newcommand{\wldnonadjcond}{
\begin{array}{l }
l < i \leq I \\
\delta(i, l) \leq d \\
c_{i-l} = \bzero \\
\end{array}
}


\begin{math} %\small
 \left. 
  \begin{array}{l}
    \textsc{Item} ~~~~ {\itembrack{[1 .. I +1], \{0,1\}^{d-1}}} \\
    %\\
    \textsc{Goal} ~~~~ {\itembrack{I+1, C}} \\
    %\\  
    \textsc{Axiom} \\ 
	\drule{}{\itembrack{1, 0^{d-1}}}{} \\
    %\\  
    \textsc{Adjacent} \\ 
	\drule{\itembrack{l, C}}{\itembrack{l + n, C \ll n}}{\wldadjcond} \\ 
	~~ \text{\small where $n = \#_1(C) + 1$} \\
%	~~ \text{\small and $\#_1(C)$ is the number of leading ones in $C$} \\
    \textsc{Non-Adjacent} \\ 
	\drule{\itembrack{l, C}}{\itembrack{l, \alpha_l^{i}(C)}}{\wldnonadjcond} \\ 
	
  \end{array} 
\right.
\end{math}


	\end{column}
	\begin{column}{0.6\linewidth}
		
	\begin{itemize}
		\item $O(Id2^{d-1})$ states
		\item $O(Id2^{d-1})$ transitions
	\end{itemize}

	\end{column}
	\end{columns}
}


\frame{
	\frametitle{Word replacement with reordering constrained by WL$2$}
	
	\only<1>{Complexity: $O(I2^{d-1})$ \\}
	\only<2>{Complexity: $O(tI2^{d-1})$ \\}	

~

\scalebox{0.8}{
\begin{tikzpicture}[->,>=stealth',shorten >=1pt,auto,node distance=2.8cm,semithick]
\tikzstyle{every state}=[draw=black,text=black]

\node[initial,state,style={initial text=}] (q10) {$\{1,2\}$};
\node[state] (q20) [above right of=q10] {$\{2,3\}$};
\node[state] (q11) [below right of=q10] {$\{1,\not2\}$};
\node[state] (q3) [right of=q11] {$\{3 \}$};
\node[state] (q21) [right of=q20] {$\{2,\not3\}$};
\node[state,accepting] (q4) [below right of=q21] {$\varnothing$};

\only<1>{
\path[draw=red] (q10) edge node {\ftext{nosso}} (q20)
	(q10) edge node {\ftext{amigo}} (q11)
	(q20) edge node {\ftext{comum}} (q21)
	(q20) edge node {\ftext{amigo}} (q3)
	(q11) edge node {\ftext{nosso}} (q3)
	(q3) edge node {\ftext{comum}} (q4)
	(q21) edge node {\ftext{amigo}} (q4);
}

\only<2>{
\path[draw=blue] (q10) edge [bend left=15] node [above] {\etext{our}} (q20)
	(q10) edge  node [below] {\etext{ours}} (q20)

	(q10) edge node [above] {\etext{friend}} (q11)
	(q10) edge [bend right=15] node [below] {\etext{mate}} (q11)

	(q20) edge [bend left=45] node [above] {\etext{ordinary}} (q21)
	(q20) edge [bend left=30] node {\etext{mutual}} (q21)
	(q20) edge node {\etext{common}} (q21)
	(q20) edge [bend right=15] node {\etext{usual}} (q21)			

	(q20) edge [bend left=15] node {\etext{friend}} (q3)
	(q20) edge [bend right=15] node {\etext{mate}} (q3)

	(q11) edge node {\etext{our}} (q3)
	(q11) edge [bend right=15] node [below] {\etext{ours}} (q3)

	(q3) edge node {\etext{usual}} (q4)
	(q3) edge [bend right=15] node {\etext{mutual}} (q4)
	(q3) edge [bend right=30] node [below] {\etext{common}} (q4)
	(q3) edge [bend right=45] node [below] {\etext{ordinary}} (q4)			

	(q21) edge [bend left=30] node [above] {\etext{friend}} (q4)
	(q21) edge node {\etext{mate}} (q4);

}

\end{tikzpicture} 
}

	
}

