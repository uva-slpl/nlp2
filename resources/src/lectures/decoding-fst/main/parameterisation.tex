\section{Parameterisation}

\frame{
	\frametitle{What are we missing?}
	
	We have characterised the set of solutions ``backed'' by our transfer model\\ \pause
	\begin{itemize}
		\item these solutions are unweighted \pause
		\item there is no obvious way to discriminate them \pause
		\item we cannot make decisions like that
	\end{itemize}
	
	\pause
	
	We are missing a parameterisation of the model
	\begin{itemize}
		\item the scoring function which will guide the decision making process
	\end{itemize}
}

\frame{
	\frametitle{Linear models}
	
	Let's call {\bf derivation}  \pause
	\begin{itemize}
		\item a translation string \pause
		\item along with any latent structure assumed by the transfer model \\
		e.g. phrase segmentation, alignment
	\end{itemize}
	\pause
	
	A linear parameterisation of the model is a function
	$$f(\mdd) = \sum_k \lambda_k H_k(\mdd)$$
	where \mdd is the derivation, and $H_k$ is one of $m$ feature functions\\
	
	~ \pause
	
	It assigns a real-valued score to each and every derivation \\
	
	~ \pause
	
	Think of it as a surrogate for translation quality at decoding time \\
	
	\hfill \citep{Berger+1996:maxent} \\
	\hfill \citep{Och+2002:maxent}
}

\frame{
	\frametitle{Feature functions}
	
	Independently capture different aspects of the translation, such as
	\begin{itemize}
		\item adequacy
		\begin{itemize}
			\item translation probabilities
			\item confidence on lexical choices
		\end{itemize}
		\item fluency
		\begin{itemize}
			\item LM probabilities
			\item confidence on reodering
		\end{itemize}
	\end{itemize}
	
	
}

\frame{
	\frametitle{Independence assumptions}
	
	Our transfer model makes independence assumptions
	\begin{itemize}
		\item ``translation happens by concatenating isolated rules''
		e.g. flat mappings, hierarchical mappings
	\end{itemize}
	\pause
	
	~
	
	Certain aspects of translation quality comply with such assumptions\looseness=-1
	\begin{itemize}
		\item how likely a certain translation rule is \\
		e.g. relative frequency in a bilingual corpus
	\end{itemize}
		
}



\frame{
	\frametitle{Structural independence: scoring rules in isolation}

\begin{textblock*}{110mm}(0.1\textwidth,0.30\textheight)

Scoring rules independently \\

~

\scalebox{0.6}{

\begin{tikzpicture}[->,>=stealth',shorten >=1pt,auto,node distance=2.8cm,semithick]
  	\tikzstyle{every state}=[draw=black,text=black]

\node[initial,state,style={initial text=}] (A) {$0$};
\node[state] (B) [right of=A] {$1$};
\node[state] (C) [right of=B] {$2$};
\node[state,accepting] (D) [right of=C] {$3$};
\only<5->{\node[state] (E) [below of=C] {$2'$};}

\only<1>{\path[color=red] (A) edge node {\ftext{nosso}} (B);}
\only<1-2>{\path[color=red] (B) edge node {\ftext{amigo}} (C);}
\only<1-3>{\path[color=red] (C) edge node {\ftext{comum}} (D);}

\only<2->{
\path[color=blue] (A) edge [bend left] node {\etext{our}/$0.6$} (B);
\path[color=blue] (A) edge [bend right] node {\etext{ours}/$0.4$} (B);
}

\only<3->{
\path[color=blue] (B) edge [bend left] node {\etext{friend}/$0.7$} (C);
\path[color=blue] (B) edge [bend right] node {\etext{mate}/$0.3$} (C);
}

\only<4->{
\path[color=blue] (C) edge [bend left=60] node {\etext{ordinary}/$0.2$} (D);
\path[color=blue] (C) edge [bend left=15] node {\etext{usual}/$0.4$} (D);
\path[color=blue] (C) edge [bend right=15] node [below]{\etext{mutual}/$0.1$} (D);
\path[color=blue] (C) edge [bend right=60] node [below] {\etext{common}/$0.3$} (D);
}

\only<5->{
\path[color=blue] (B) edge [bend right] node [below] {\etext{mutual}} (E);
\path[color=blue] (E) edge [bend right] node [below] {\etext{friend}/$0.8$} (D);

\path[color=blue] (B) edge [bend left=90] node [above] {\etext{camarada}/$0.2$} (D);

}

\end{tikzpicture} 

}

~


\only<6->{
\hfill inference runs in time linear with the size of the automaton
}

\end{textblock*}

	
}


\frame{
	\frametitle{Independence assumptions}
	
	Our transfer model makes independence assumptions
	\begin{itemize}
		\item ``translation happens by concatenating isolated rules''
		e.g. flat mappings, hierarchical mappings
	\end{itemize}
	
	
	~
	
	{\color{gray}
	Certain aspects of translation quality comply with such assumptions\looseness=-1
	\begin{itemize}\color{gray}
		\item how likely a certain translation rule is \\
		e.g. relative frequency in a bilingual corpus
	\end{itemize}
	}
	
	~
	
	\pause
	Certain aspects do not comply with such assumptions
	\begin{itemize}
		\item fluency as captured by an $n$-gram LM component
	\end{itemize}
	
}

\frame{
	\frametitle{Scoring strings with a $2$-gram LM}

\begin{textblock*}{110mm}(0.1\textwidth,0.10\textheight)

\scalebox{0.6}{

\begin{tikzpicture}[->,>=stealth',shorten >=1pt,auto,node distance=2.8cm,semithick]
  	\tikzstyle{every state}=[draw=black,text=black]

\node[initial,state,style={initial text=}] (A) {$0$};
\node[state] (B) [right of=A] {$1$};
\node[state] (C) [right of=B] {$2$};
\node[state] (D) [right of=C] {$3$};
\node[state,accepting] (E) [right of=D] {$4$};
%\node[state] (E) [below of=C] {$2'$};

\only<1,6->{\path[color=blue] (A) edge [bend left] node {\etext{our}/$0.6$} (B);}
\only<2-5>{\path[color=red] (A) edge [bend left] node {\etext{our}/$0.6$} (B);}
\only<1,2,4->{\path[color=blue] (A) edge [bend right] node {\etext{ours}/$0.4$} (B);}
\only<3>{\path[color=red] (A) edge [bend right] node {\etext{ours}/$0.4$} (B);}


\only<1-3,5->{
\path[color=blue] (B) edge [bend right] node {\etext{mate}/$0.3$} (C);}
\only<4>{\path[color=red] (B) edge [bend right] node {\etext{mate}/$0.3$} (C);}

\only<1-4,10->{
\path[color=blue] (B) edge [bend left] node {\etext{friend}/$0.7$} (C);}
\only<5-9>{\path[color=red] (B) edge [bend left] node {\etext{friend}/$0.7$} (C);}


\only<1-5,7->{\path[color=blue] (C) edge [bend left=60] node {\etext{ordinary}/$0.2$} (D);}
\only<6,10>{\path[color=red] (C) edge [bend left=60] node {\etext{ordinary}/$0.2$} (D);}

\only<1-6,8->{\path[color=blue] (C) edge [bend left=15] node {\etext{usual}/$0.4$} (D);}
\only<7,11>{\path[color=red] (C) edge [bend left=15] node {\etext{usual}/$0.4$} (D);}

\only<1-7,9->{\path[color=blue] (C) edge [bend right=15] node [below]{\etext{mutual}/$0.1$} (D);}
\only<8,12>{\path[color=red] (C) edge [bend right=15] node [below]{\etext{mutual}/$0.1$} (D);}

\only<1-8,10->{\path[color=blue] (C) edge [bend right=60] node [below] {\etext{common}/$0.3$} (D);}
\only<9,13>{\path[color=red] (C) edge [bend right=60] node [below] {\etext{common}/$0.3$} (D);}

\only<1-9>{\path[color=blue] (D) edge node {\eos} (E);}
\only<10->{\path[color=red] (D) edge node {\eos} (E);}


%\path[color=blue] (B) edge [bend right] node [below] {\etext{mutual}} (E);
%\path[color=blue] (E) edge [bend right] node [below] {\etext{friend}/$0.8$} (D);

%\path[color=blue] (B) edge [bend left=90] node [above] {\etext{camarada}/$0.2$} (D);


\end{tikzpicture} 

}
\end{textblock*}


\begin{textblock*}{110mm}(0.1\textwidth,0.40\textheight)


\scalebox{0.6}{

\begin{tikzpicture}[->,>=stealth',shorten >=1pt,auto,node distance=2.8cm,semithick]
  	\tikzstyle{every state}=[draw=black,text=black]

\only<2->{\node[initial,state,style={initial text=}] (q0) at (0, 0) {\bos};}
\only<2->{\node[state] (qOur) at (3, 1) {o};}
\only<3->{\node[state,dotted] (qOurs) at (3, -2) {os};}
\only<4->{\node[state,dotted] (qMate) at (5, 3) {m};}
\only<5->{\node[state] (qFriend) at (5, -1) {f};}

\only<6->{\node[state] (qOrdinary) at (10, 3) {or};}
\only<7->{\node[state] (qUsual) at (10, 1) {u};}
\only<8->{\node[state] (qMutual) at (10, -1) {mu};}
\only<9->{\node[state] (qCommon) at (10, -3) {co};}
\only<10->{\node[state,accepting] (qF) at (15, 0) {\etext{\eos}};}

\only<2->{\path[color=blue] (q0) edge node {\etext{our}/$0.6 \otimes \textcolor{Red}{0.06}$} (qOur);}

\only<3->{\path[color=blue] (q0) edge node [below] {\etext{ours}/$0.4 \otimes \textcolor{Red}{0.03}$} (qOurs);}

\only<4->{\path[color=blue] (qOur) edge node {\etext{mate}/$0.3 \otimes \textcolor{Red}{0.02}$} (qMate);}

\only<5->{\path[color=blue] (qOur) edge node [below] {\etext{friend}/$0.7 \otimes \textcolor{Red}{0.05}$} (qFriend);}

\only<6->{\path[color=blue] (qFriend) edge [bend left] node [below] {\etext{ordinary}/$0.2 \otimes \textcolor{Red}{0.0005}$} (qOrdinary);}

\only<7->{\path[color=blue] (qFriend) edge [bend left=15] node [below]{\etext{usual}/$0.4 \otimes \textcolor{Red}{0.002}$} (qUsual);}

\only<8->{\path[color=blue] (qFriend) edge node [below] {\etext{mutual}/$0.1 \otimes \textcolor{Red}{0.002}$} (qMutual);}

\only<9->{\path[color=blue] (qFriend) edge node [below] {\etext{common}/$0.32 \otimes \textcolor{Red}{0.001}$} (qCommon);}

\only<10->{\path[color=blue] (qOrdinary) edge node  {\etext{\eos}/$\textcolor{Red}{0.016}$} (qF);}

\only<11->{\path[color=blue] (qUsual) edge node  {\etext{\eos}/$\textcolor{Red}{0.017}$} (qF);}

\only<12->{\path[color=blue] (qMutual) edge node  {\etext{\eos}/$\textcolor{Red}{0.0022}$} (qF);}

\only<13->{\path[color=blue] (qCommon) edge node  {\etext{\eos}/$\textcolor{Red}{0.024}$} (qF);}

%\only<10->{
%\path[color=blue] (qOrdinary) edge node {\eos} (qF);
%\path[color=blue] (qUsual) edge node {\eos} (qF);
%\path[color=blue] (qMutual) edge node {\eos} (qF);
%\path[color=blue] (qCommon) edge node {\eos} (qF);
%}

\end{tikzpicture} 

}

~

\hfill requires unpacking the representation
\end{textblock*}

	
}


\frame{
	\frametitle{Scoring whole sentences}

	Imagine a feature function that requires a complete translation
	\pause
	\begin{itemize}		
		\item unbounded LM \\
		e.g. via suffix arrays \citep{Zhang+2006:SALM}\\
		e.g. via RNN language model
		\item estimated overall translation quality
	\end{itemize}
	\pause
	No factorisation at the phrase (nor $n$-gram) level \pause
	\begin{itemize}
		\item requires fully unpacking the representation \pause
		\item making dependencies explicit through the graphical structure
	\end{itemize}
	
	
}

\frame{

	\frametitle{Scoring whole sentences: example}

\scalebox{0.6}{

\begin{tikzpicture}[->,>=stealth',shorten >=1pt,auto,node distance=2.8cm,semithick]
  	\tikzstyle{every state}=[draw=black,text=black]
  	\tikzstyle{every path}=[draw=blue,text=blue]	

\node[initial,state,style={initial text=}] (q0) {$0$};
\node[state] (q1) [above right of=q0] {$1$};
\node[state] (q2) [right of=q1] {$2$};
\node[state] (q3) [right of=q2] {$3$};
\node[state] (q4) [right of=q3] {$4$};
\node[state,accepting] [right of=q4] (q5) {$5$};

\node[state] (q6) [right of=q0] {$6$};
\node[state] (q7) [right of=q6] {$7$};
\node[state] (q8) [right of=q7] {$8$};
\node[state] (q9) [right of=q8] {$9$};
\node[state,accepting] [right of=q9] (q10) {$10$};

\node[state] (q11) [below right of=q0] {$11$};
\node[state] (q12) [right of=q11] {$12$};
\node[state] (q13) [right of=q12] {$13$};
\node[state] (q14) [right of=q13] {$14$};
\node[state,accepting] [right of=q14] (q15) {$15$};

\node[state] (q16) [below of=q11] {$16$};
\node[state] (q17) [right of=q16] {$17$};

\path
(q0) edge node {\etext{one}} (q1)
(q1) edge node {\etext{story}} (q2)
(q2) edge node {\etext{'s}} (q3)
(q3) edge node {\etext{two}} (q4)
(q4) edge node {\etext{villages}/$w_1$} (q5);

\path
(q0) edge node {\etext{one}} (q6)
(q6) edge node {\etext{story}} (q7)
(q7) edge node {\etext{'s}} (q8)
(q8) edge node {\etext{two}} (q9)
(q9) edge node {\etext{cities}/$w_2$} (q10);

\path
(q0) edge node {\etext{one}} (q11)
(q11) edge node {\etext{story}} (q12)
(q12) edge node {\etext{'s}} (q13)
(q13) edge node {\etext{two}} (q14)
(q14) edge node {\etext{towns}/$w_3$} (q15);

\path
(q0) edge node {\etext{...}} (q16)
(q16) edge node {\etext{...}} (q17);
			
\end{tikzpicture} 
}

~

	Exhaustive enumeration 
	%\begin{itemize}
%		\item number of edges exponential with input length%
%		\item \alert{intractable}
% 	\end{itemize}

}

\frame{
	\frametitle{Not all is lost}
	
	Most features we can reliably estimate \pause
	\begin{itemize}
		\item are rarely sensitive to global context \pause
		\item are quite incremental 
	\end{itemize}
	\pause
	
	$n$-gram LMs are good examples
	
	\begin{itemize}
		\item there are up to $|\Delta|^{n-1}$ contexts that must be made explicit \pause
		\item nodes must group derivations sharing the same context \pause
		\item polynomial, though often prohibitive (\alert{impracticable})

	\end{itemize}

}

\frame{
	\frametitle{Recap 4}
	
	\begin{enumerate}
		\item a characterisation the space of solutions \pause
		\item a linear parameterisation of the model \pause
		\item impact of parameterisation on packed representations
	\end{enumerate}
	\pause
	
	What's left? \pause
	\begin{itemize}
		\item more examples of models and impact on representation
		\begin{itemize}
			\item distance-based reordering
			\item lexicalised models
			\item a global feature function
		\end{itemize}
		\item inference algorithms \pause
		\item techniques to make inference feasible for interesting models	
	\end{itemize}
}

