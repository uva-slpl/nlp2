
%\documentclass[xcolor=pdftex,dvipsnames,table]{beamer}
\documentclass[xcolor=pdftex,dvipsnames,table]{beamer}

\usetheme{Amsterdam}

\usefonttheme[onlymath]{serif}
\setbeamertemplate{navigation symbols}{}
\setbeamertemplate{footline}[frame number]


\usepackage{lmodern}
\usepackage{comment}
\usepackage{natbib}
\usepackage{graphicx}
%\usepackage{pgf}
\usepackage{pbox}
\usepackage{pifont}
\usepackage{alltt}
\usepackage{verbatim}
\usepackage{multirow}
\usepackage{xspace}
\usepackage{cases}
\usepackage{geometry}
%\usepackage{tikz}
%\usetikzlibrary{arrows,automata,positioning}
%\usepackage[absolute,overlay]{textpos}
\usepackage[normalem]{ulem}
%\usepackage{tikz-qtree}
\usepackage{pbox}
%\usepackage{ragged2e}
%\usepackage{pifont}

%\usepackage[all]{xy}

\usepackage{latexsym}
\usepackage{amsmath}
\usepackage{amssymb}
\usepackage{xfrac}

\usepackage{notation}
\usepackage{variables}
\usepackage{drawing}
%\usepackage{xparse}
%\usepackage{tikz}
%\usetikzlibrary{calc}



\newcommand{\dquote}[1]{``{#1}''}
\newcommand{\squote}[1]{`{#1}'}
\newcommand{\pgivenbf}[3]{${#1}(\mathbf{#2} | \mathbf{#3})$}
\newcommand{\pgiven}[3]{${#1}({#2} | {#3})$}
\newcommand{\pofbf}[2]{${#1}(\mathbf{#2})$}
\newcommand{\pof}[2]{${#1}({#2})$}
\newcommand{\indice}[1]{$_{#1}$}
\newcommand{\cgray}[1]{\textcolor{gray}{{#1}}}
\newcommand{\cblue}[1]{\textcolor{blue}{{#1}}}
\newcommand{\cred}[1]{\textcolor{red}{{#1}}}
\newcommand{\coran}[1]{\textcolor{Orange}{{#1}}}
\newcommand{\cgreen}[1]{\textcolor{Green}{{#1}}}
\newcommand{\cellbl}{\cellcolor{black}}
\newcommand{\cellblue}{\cellcolor{blue}}
\newcommand{\cellgreen}{\cellcolor{green}}
\newcommand{\cellr}{\cellcolor{red}}
\newcommand{\cellg}{\cellcolor{gray}}
\newcommand{\celldg}{\cellcolor{darkgray}}
\newcommand{\lra}{$\leftrightarrow$}
\newcommand{\WX}{\textcolor{white}{X}}
\newcommand{\WDot}{\textcolor{white}{$\cdot$}}
\newcommand{\WN}[1]{\textcolor{white}{#1}}
\newcommand{\vtext}[1]{\begin{sideways}#1\end{sideways}}
\newcommand{\phr}[1]{$\overset{\_}{#1}$}
\newcommand{\phs}{\overset{\_}{s}}
\newcommand{\pht}{\overset{\_}{t}}
\newcommand{\hypoe}[2]{\fbox{$\overset{\text{#1}}{#2}$}}
\newcommand{\hypot}[2]{\fbox{$\overset{\text{#1}}{\text{#2}}$}}
%\newcommand{\argmax}{\operatornamewithlimits{argmax}}
%\newcommand{\thickbar}[1]{\mathbf{\bar{\text{$#1$}}}}



% declares a document
\begin{document}



	%\title{Employee's social media use}
	%\title{Social media use by employees}
	\title{Phrase-based SMT}
	%\subtitle{for unsupervised language learning}

	\author{Miguel Rios}
	\institute[UvA]{
		%\inst{1}
		Universiteit van Amsterdam\\
	}

	\date{\today}
	
	% Title page
	{\setbeamertemplate{footline}{}
	\begin{frame}[plain]
		\titlepage
	\end{frame}
	}


	% Table of contents	
	%\frame[allowframebreaks]{
	{\setbeamertemplate{footline}{}
	\begin{frame}
		\frametitle{Content}
		\tableofcontents
	\end{frame}
	}



	% trick to start counting from the table of contents
	\setcounter{framenumber}{0}


	% SLIDES
	%\section{Word-based SMT}
\subsection{IBM models 1 and 2}
\frame{
    \frametitle{The Noisy-Channel approach}
	
	Bayes rule 
	
	$$P(E|F) = \frac{P(E)P(F|E)}{P(F)}$$
	
	Inference
	
	$$\hat{E} = \argmax_E P(E)P(F|E)$$
	
	Estimation
	
	\begin{itemize}
		\item $P(E)$ $n$-gram LM
		\item $P(F|E)$ TM
	\end{itemize}
	
}

\frame{
    \frametitle{The IBM models}

		$$P(F|E) = \sum_A P(A,F|E) $$
	\begin{center}

    	 \includegraphics[width=0.8\textwidth]{"img/wilker-align"}
	\end{center}
}



\frame{
    \frametitle{Decoding with models 1 \& 2?}
	\begin{center}
    	 \includegraphics[width=0.8\textwidth]{"img/wilker-align"}
	\end{center}
	
	~
	\pause
	how to explain insertions on the English side?
}






	\section{Introduction}

\frame{
	\frametitle{Recap}
	We looked into Alignment a directional word-based model.
	\begin{itemize}
	\item Parametrisation: Categorical.
	\item Estimation techniques: EM vs VB.
	\end{itemize}
	\pause
	We have not look into generation:
	\begin{itemize}
	\item No model of length
 	\item No model of segmentation
 	\item Bad model for translation
	\end{itemize}

}

\frame{
	\frametitle{Translation}

	Model:

	$$P(E|F) = \frac{P(E)P(F|E)}{P(F)}$$

	Prediction:

	$$\hat{E} = \argmax_E P(E)P(F=f|E)$$

	Estimation:

	\begin{itemize}
		\item $P(E)$ $n$-gram LM.
		\item $P(F|E)$ TM.
	\end{itemize}

}

\frame{
	\frametitle{Word-based SMT}
	\citep{Brown+1993:smt}\\
	\begin{center}
	\includegraphics[scale=0.5]{"img/wbmt"}
	\end{center}
	\begin{footnotesize}
	Figure: \citet{Koehn:2010:SMT}
	\end{footnotesize}
	

}

\frame{
	\frametitle{Limitations of word-based approach}

	Linguistically
	\begin{itemize}
		\item Can not translate many-to-one or many-to-many
		\item Compositionality of translation\\
			multi-word / idiomatic expressions.
	\end{itemize}

	~

	Computationally during prediction
	\begin{itemize}
		\item $n!$ permutations in decoding.
	\end{itemize}
}


\frame{
	\frametitle{Phrase-based model}
	Change of units: phrase.\\
	\begin{center}
	\includegraphics[scale=0.5]{"img/phrase-model-alignment"}
	\end{center}
	\begin{footnotesize}
	Figure: \citet{Koehn:2010:SMT}
	\end{footnotesize}
}


\frame{
	\frametitle{Phrase-based model}

	Phrase pairs as translation units
	\begin{itemize}
		\item Capture non-compositional translations.
		\item Exploit (local) reordering patterns.
	\end{itemize}


}

\frame{
	\frametitle{Illustration}
	\includegraphics[scale=0.5]{"img/PB extraction 2"}

	\pause
	~
	\begin{center}
	\begin{tabular}{l  r}

		J'$_1$ ai$_2$ les$_3$ yeux$_4$ noirs$_5$ & input\\ \pause
		{[\textcolor{blue}{J'$_1$ ai$_2$}] [\textcolor{Green}{les$_3$ yeux$_4$}] [\textcolor{red}{noirs$_5$}]} & segmentation\\ \pause
		{[\textcolor{blue}{J'$_1$ ai$_2$}]$_1$ [\textcolor{red}{noirs$_5$}]$_3$ [\textcolor{Green}{les$_3$ yeux$_4$}]$_2$} & ordering\\		\pause
		{[\textcolor{blue}{I have}]$_1$ [\textcolor{red}{black}]$_3$ [\textcolor{Green}{eyes}]$_2$} & translation \\	\pause
		 &\textbf{Derivation}
	\end{tabular}
	\end{center}
}



\section{Model}
\frame{
    \frametitle{Modelling Derivations}

	\begin{align*}
		P(e,d|f) = \frac{\exp(S_{\theta }(e,d,f))}{\sum_{{e}'}\sum_{{d}'} \exp(S_{\theta}({e}',{d}', f))}
	\end{align*}
	\pause
	Challenging normalisation.\\
	Large space of derivations:\\
	\begin{itemize}
	\item Number of segments.
	\item Number of permutations.
	\item Number of translations.
	\end{itemize}
}

\frame{
    \frametitle{Discriminative classifier}
	\begin{itemize}
	\item Give up on marginalisation of $d$
	\item Give up on probabilistic modelling
	\item How?\pause
	\item If we look at the prediction:
	\end{itemize}
	\begin{align*}
	\hat{e},\hat{d} &= \argmax_{e,d|f} \log P(e,d|f)\\
	&= \argmax_{e,d|f} S_{\theta}(e,d,f) - \underbrace{\log \sum_{{e}'}\sum_{{d}'} \exp(S_{\theta}({e}',{d}', f))}_{\text{constant for any} (e,d | f)}\\
	&= \argmax_{e,d|f} S_{\theta}(e,d,f)
	\end{align*}
	Trained discriminatively (e.g. structured perceptron).
}




\frame{
	\frametitle{Linear model}

	The score function$S_{\theta}$ is defined as a linear model.\\
	$$S_{\theta}(e,d,f) = \theta^T H(e,d,f)$$
	where $\theta$ are parameters\\
	$h$ are feature functions.\\
	\pause
	Linear model decomposes over phrases.\\
	$$S_{\theta}(e,d,f) = \theta^T \sum_{i}^{n} \underbrace{h_i(d_i|e,f)}_{\text{local feature function}}$$
	Model featurises steps in the derivation independently.

}

\frame{
    \frametitle{PBSMT Model}
   
     \begin{itemize}
    \item Feature functions $n=3$
    \item Translation feature function: $$h_1 = \log  P(\bar{f}| \bar{e})$$
    \item Language Model feature function: $$h_2 = \log  P(e|e_{\text{past}})$$
    \item Distortion feature function: $$h_3 = \log  d(\text{start}_k - \text{end}_{k-1} - 1)$$
    \end{itemize}

}


\frame{
	\frametitle{Phrase pairs from word alignments}

	%\citet{Koehn+2003:PBSMT}
	\only<1>{
		\includegraphics[scale=0.5]{"img/PB extraction 0"}
	}
	\only<2>{
		\includegraphics[scale=0.5]{"img/PB extraction 1"}
	}
	\only<3>{
		\includegraphics[scale=0.5]{"img/PB extraction 1b"}
	}
	\only<4>{
		\includegraphics[scale=0.5]{"img/PB extraction 2"}
	}
	\only<5>{
		\includegraphics[scale=0.5]{"img/PB extraction 2b"}
	}
	\only<6>{
		\includegraphics[scale=0.5]{"img/PB extraction 3"}
	}
	\only<7>{
		\includegraphics[scale=0.5]{"img/PB extraction 4"}
	}
	\only<8>{
		\includegraphics[scale=0.5]{"img/PB extraction 5"}
	}
	\only<9>{
		\includegraphics[scale=0.5]{"img/PB extraction all"}
	}


	\begin{itemize}
		\item<2-> multiple derivations can explain an ``observed'' phrase pair \\
		\item<9> we extract all of them once, irrespective of derivation
	\end{itemize}
}


\frame{
    \frametitle{Phrase Table}
    \begin{itemize}
    \item Goal: Learn phrase translation table from parallel corpus.
	\pause
    \item Three stages:
    \item Word alignment given IBM.
    \item Extraction of phrase pairs.
    \item Phrase scoring.

    \end{itemize}

}


\frame{
	\frametitle{Phrase extraction}
				Let $(\bar{f},\bar{e})$ be a phrase pair\\
				Let $A$ be an alignment matrix\\
				\pause
				\begin{block}{$(\bar{f},\bar{e})$ consistent with $A$ if, and only if:}
					\begin{itemize}
						\pause
						\item Words in $\bar{f}$, if aligned, align only with words in $\bar{e}$\\
						\pause
						\begin{tiny}
						\begin{columns}
						\begin{column}{1cm}
						\begin{tabular}{|p{0.1cm}|p{0.1cm}|p{0.1cm}|}
							\multicolumn{3}{c}{\cblue{C}} \\ \hline
							\cellg $\bullet$ & \cellg & \cellg \\ \hline
							\cellg & \cellg $\bullet$ & \cellg $\bullet$ \\ \hline
							 &  & \\ \hline
						\end{tabular}
						\end{column}
						\begin{column}{1cm}
						\begin{tabular}{|p{0.1cm}|p{0.1cm}|p{0.1cm}|}
							\multicolumn{3}{c}{\cblue{C}} \\ \hline
							\cellg $\bullet$ & \cellg & \cellg \\ \hline
							\cellg & \cellg $\bullet$ & \cellg $\bullet$ \\ \hline
							\cellg & \cellg & \cellg \\ \hline
						\end{tabular}
						\end{column}
						\begin{column}{1cm}
						\begin{tabular}{|p{0.1cm}|p{0.1cm}|p{0.1cm}|}
							\multicolumn{3}{c}{\cred{I}} \\ \hline
							\cellg $\bullet$ & \cellg & \\ \hline
							\cellg & \cellg $\bullet$ & \textcolor{red}{$\bullet$} \\ \hline
							 &  & \\ \hline
						\end{tabular}
						\end{column}
						\end{columns}
						\end{tiny}

						\pause
						\item Words in $\bar{e}$, if aligned, align only with words in $\bar{f}$\\
						\pause
						\begin{tiny}
						\begin{columns}
						\begin{column}{1cm}
						\begin{tabular}{|p{0.1cm}|p{0.1cm}|p{0.1cm}|}
							\multicolumn{3}{c}{\cblue{C}} \\ \hline
							\cellg $\bullet$ & \cellg & \\ \hline
							\cellg & \cellg $\bullet$ & \\ \hline
							\cellg & \cellg $\bullet$ & \\ \hline
						\end{tabular}
						\end{column}
						\begin{column}{1cm}
						\begin{tabular}{|p{0.1cm}|p{0.1cm}|p{0.1cm}|}
							\multicolumn{3}{c}{\cblue{C}} \\ \hline
							\cellg $\bullet$ & \cellg & \cellg \\ \hline
							\cellg & \cellg $\bullet$ & \cellg \\ \hline
							\cellg & \cellg $\bullet$ & \cellg \\ \hline
						\end{tabular}
						\end{column}
						\begin{column}{1cm}
						\begin{tabular}{|p{0.1cm}|p{0.1cm}|p{0.1cm}|}
							\multicolumn{3}{c}{\cred{I}} \\ \hline
							\cellg $\bullet$ & \cellg & \\ \hline
							\cellg & \cellg $\bullet$ & \\ \hline
							 & \textcolor{red}{$\bullet$} & \\ \hline
						\end{tabular}
						\end{column}
						\end{columns}
						\end{tiny}

						\pause
						\item $(\bar{f},\bar{e})$ must contain at least one alignment point\\
						\pause
						\begin{tiny}
						\begin{columns}
						\begin{column}{1cm}
						\begin{tabular}{|p{0.1cm}|p{0.1cm}|p{0.1cm}|}
							\multicolumn{3}{c}{\cblue{C}} \\ \hline
							\cellg $\bullet$ & \cellg & \cellg \\ \hline
							\cellg  &\cellg $\bullet$ & \cellg \\ \hline
							\cellg  & \cellg & \cellg \\ \hline
						\end{tabular}
						\end{column}
						\begin{column}{1cm}
						\begin{tabular}{|p{0.1cm}|p{0.1cm}|p{0.1cm}|}
							\multicolumn{3}{c}{\cblue{C}} \\ \hline
							\cellg $\bullet$ & & \\ \hline
							  & \cellg $\bullet$ & \cellg \\ \hline
							 & \cellg & \cellg \\ \hline
						\end{tabular}
						\end{column}
						\begin{column}{1cm}
						\begin{tabular}{|p{0.1cm}|p{0.1cm}|p{0.1cm}|}
							\multicolumn{3}{c}{\cred{I}} \\ \hline
							$\bullet$ &  & \\ \hline
							 & $\bullet$ & \\ \hline
							 &  & \cellg \\ \hline
						\end{tabular}
						\end{column}
						\end{columns}
						\end{tiny}

					\end{itemize}
				\end{block}

}

\frame{
	\frametitle{Feature Translation Model}
	Features $$\log  P(\bar{f}, \bar{e})$$ and $$\log  P(\bar{e}, \bar{f})$$\\
	Number of times a (consistent) phrase pair is ``observed''\\
	$$c(\bar{f}, \bar{e})$$

	Relative frequency counting
	$$\varphi(\bar{f}|\bar{e}) = \frac{c(\bar{f}, \bar{e})}{\sum_{\bar{f}'} c(\bar{f}', \bar{e})}$$
	
	

}


\frame{
	\frametitle{Feature Distortion}
	Feature $$h_3 = \log  d(\text{start}_k - \text{end}_{k-1} - 1)$$\\
	Example

	\begin{center}
	\begin{tabular}{| l | l | p{1cm} | p{1cm} | p{1cm} | p{1cm} |}
	\hline
	& & \textcolor{red}{I} & \textcolor{red}{have} & \textcolor{red}{black} & \textcolor{red}{eyes} \\ \hline
	\textcolor{gray}{1}& \textcolor{blue}{J'} & \multicolumn{2}{c|}{\multirow{2}{*}{1}}  & & \\ \cline{1-2}\cline{5-6}
	\textcolor{gray}{2}& \textcolor{blue}{ai} & \multicolumn{2}{c|}{} & & \\ \hline
	\textcolor{gray}{3}& \textcolor{blue}{les} & & & & \multicolumn{1}{c|}{\multirow{2}{*}{3}} \\ \cline{1-5}
	\textcolor{gray}{4}& \textcolor{blue}{yeux} & & & &  \\ \hline
	\textcolor{gray}{5}& \textcolor{blue}{noirs} & & & \multicolumn{1}{c|}{2} & \\ \hline
	\end{tabular}
	\end{center}

	\begin{columns}

	\begin{column}{0.3\textwidth}
	\begin{itemize}
		\item $\bar{f}_1 = \textcolor{blue}{\text{J' ai}}$
		\item $\bar{e}_1 = \textcolor{red}{\text{I have}}$
		\item $\text{start}_1 = 1$
		\item $\text{end}_1 = 2$
	\end{itemize}
	\end{column}
	\begin{column}{0.3\textwidth}
	\begin{itemize}
		\item $\bar{f}_2 = \textcolor{blue}{\text{noirs}}$
		\item $\bar{e}_2 = \textcolor{red}{\text{black}}$
		\item $\text{start}_2 = 5$
		\item $\text{end}_2 = 5$
	\end{itemize}
	\end{column}
	\begin{column}{0.3\textwidth}
	\begin{itemize}
		\item $\bar{f}_3 = \textcolor{blue}{\text{les yeux}}$
		\item $\bar{e}_3 = \textcolor{red}{\text{eyes}}$
		\item $\text{start}_3 = 3$
		\item $\text{end}_3 = 4$
	\end{itemize}
	\end{column}

	\end{columns}
}

\frame{
	\frametitle{Feature Language Model}
	Feature n-gram language model $$\log  P(e|e_{\text{past}})$$\\
	Estimated independently on monolingual data.
	\begin{center}
	\includegraphics[scale=0.3]{"img/Unigram"}
	\end{center}
	
	\begin{footnotesize}
	http://recognize-speech.com/images/Antonio/Unigram.png
	\end{footnotesize}
}

%\frame{
%	\frametitle{Feature Segmentation}


%}




\section{Prediction}





\frame{
	\frametitle{Decoding}
	\begin{center}
	\includegraphics[scale=0.3]{"img/beam_search"}
	\end{center}
	\begin{footnotesize}
	Figure: \citet{Koehn:2010:SMT}
	\end{footnotesize}
	

}


\frame{
	\frametitle{Translation Options}

	\begin{itemize}
	\item Europarl phrase table: 2727 matching phrase pairs for a sentence.
	\item Search problem with beam search:
	\begin{enumerate}
	\item From phrase translation table for all input phrases.
	\item Initial hypothesis: no input words covered, no output produced.
	\item Pick any translation option, create new hypothesis.
	%\item Create hypotheses for all other translation options
	\item Expand hypotheses from created partial hypothesis.
	\item Backtrack from highest scoring complete hypothesis.
	\end{enumerate}
	\end{itemize}

}

	
	%\section{Decoding}
\subsection{Complexity}
\frame{
        \frametitle{Decoding}

        Disambiguation problem
        \begin{align*}
                \hat{E} 
                &= \argmax_E P(E)P(F|E) \\
                &= \argmax_E P(E) \sum_A P(F,A|E)
        \end{align*}
        {\small \hfill NP-complete \citep{Simaan:2002:complexity}}

        \pause

        ~

        Viterbi approximation
        \begin{align*}
                \hat{E} 
                &\approx \argmax_{E, A} P(E) P(F,A|E)\\
        \end{align*}

}

\frame{
	\frametitle{Viterbi decoding}
	The alignment space (or space of \emph{derivations})
	\begin{itemize}
		\item $O(2^n)$ segmentations\\
		\item $O(n!)$ permutations\\
		\item $O(t^n)$ substitutions\\
	\end{itemize}
	\pause
	~
	
	Packed representation using finite-state transducers
	$$O(n^2 \times \alert{2^n} \times t)$$
	\hfill NP-complete (TSP) \citep{Knight:1999:tsp,Zaslavskiy+2009:tsp} 
	
	
}

\frame{
	\frametitle{Complete model}
	
	\begin{align*}
		\alert{P(E)}P(F,S|E) 
		&= \alert{\prod_{j=1}^{|E|} \psi(e_j|e_{j - n + 1}^{j - 1})} \prod_{i=1}^{|S|} \textcolor{blue}{\phi(\bar{f}_i|\bar{e}_i)} \textcolor{Green}{\delta(\text{start}_i - \text{end}_{i-1} - 1)}
	\end{align*}

	Approximations:
	\begin{itemize}
		\item distortion limit $d$: $2^n \to 2^d$
		\item maximum phrase length $m$: $n^2 \to n \times m$
	\end{itemize}
	
	~

	\begin{itemize}

		\item alignment space $O(\textcolor{Green}{2^d} \times \textcolor{blue}{n \times m}\times t )$
		\item weighted derivations $O(\textcolor{Green}{2^d} \times \textcolor{blue}{n \times m} \times t \times \alert{|\Delta|^{k-1}})$ \\
		where $P(E)$ is a $k$-gram LM components over $\Delta^*$\\
		and $|\Delta| \propto t \times n$
	\end{itemize}

	\only<2->{
	\textbf<2->{This space is too large for exact inference}
	\begin{itemize}
		\item<3> pruning: beam search
	\end{itemize}
	}
	
}

\frame{
    \frametitle{Complexity}
	 \citep{Knight:1999:tsp}
	\begin{center}
       \only<1>{
                \includegraphics[width=0.7\textwidth]{"img/knight-tsp1"}
        }
        \only<2>{
                \includegraphics[width=0.5\textwidth]{"img/knight-tsp2"}
        }
	\end{center}
}

	
	\newcounter{finalframe}
	\setcounter{finalframe}{\value{framenumber}}
	
	
	{\setbeamertemplate{footline}{}
    \begin{frame}[plain]{Questions?}
    \end{frame}
  	}
	
	\section*{References}
	
	%\input{backup}


	\frame[allowframebreaks]{ \frametitle{References}
        \bibliographystyle{plainnat}
        \bibliography{../bib}
	}

	

	% Trick to discount regular frames from the total of backup frames
	\setcounter{framenumber}{\value{finalframe}}
	
	
\end{document}
