
%\documentclass[xcolor=pdftex,dvipsnames,table]{beamer}
\documentclass[xcolor=pdftex,dvipsnames,table]{beamer}

\usetheme{Amsterdam}

\usefonttheme[onlymath]{serif}
\setbeamertemplate{navigation symbols}{}
\setbeamertemplate{footline}[frame number]
\usepackage{tikz}
\usetikzlibrary{bayesnet}
\usepackage{animate}
\usepackage{lmodern}
\usepackage{comment}
\usepackage{natbib}
\usepackage{graphicx}
%\usepackage{pgf}
\usepackage{pbox}
\usepackage{hyperref}
\usepackage{pifont}
\usepackage{alltt}
\usepackage{verbatim}
\usepackage{multirow}
\usepackage{xspace}
\usepackage{cases}
\usepackage{geometry}
%\usepackage{tikz}
%\usetikzlibrary{arrows,automata,positioning}
%\usepackage[absolute,overlay]{textpos}
\usepackage[normalem]{ulem}
%\usepackage{tikz-qtree}
\usepackage{pbox}
%\usepackage{ragged2e}
%\usepackage{pifont}

%\usepackage[all]{xy}

\usepackage{latexsym}
\usepackage{amsmath}
\usepackage{amssymb}
\usepackage{xfrac}

\usepackage{notation}
\usepackage{variables}
\usepackage{drawing}
%\usepackage{xparse}
%\usepackage{tikz}
%\usetikzlibrary{calc}



\newcommand{\dquote}[1]{``{#1}''}
\newcommand{\squote}[1]{`{#1}'}
\newcommand{\pgivenbf}[3]{${#1}(\mathbf{#2} | \mathbf{#3})$}
\newcommand{\pgiven}[3]{${#1}({#2} | {#3})$}
\newcommand{\pofbf}[2]{${#1}(\mathbf{#2})$}
\newcommand{\pof}[2]{${#1}({#2})$}
\newcommand{\indice}[1]{$_{#1}$}
\newcommand{\cgray}[1]{\textcolor{gray}{{#1}}}
\newcommand{\cblue}[1]{\textcolor{blue}{{#1}}}
\newcommand{\cred}[1]{\textcolor{red}{{#1}}}
\newcommand{\coran}[1]{\textcolor{Orange}{{#1}}}
\newcommand{\cgreen}[1]{\textcolor{Green}{{#1}}}
\newcommand{\cellbl}{\cellcolor{black}}
\newcommand{\cellblue}{\cellcolor{blue}}
\newcommand{\cellgreen}{\cellcolor{green}}
\newcommand{\cellr}{\cellcolor{red}}
\newcommand{\cellg}{\cellcolor{gray}}
\newcommand{\celldg}{\cellcolor{darkgray}}
\newcommand{\lra}{$\leftrightarrow$}
\newcommand{\WX}{\textcolor{white}{X}}
\newcommand{\WDot}{\textcolor{white}{$\cdot$}}
\newcommand{\WN}[1]{\textcolor{white}{#1}}
\newcommand{\vtext}[1]{\begin{sideways}#1\end{sideways}}
\newcommand{\phr}[1]{$\overset{\_}{#1}$}
\newcommand{\phs}{\overset{\_}{s}}
\newcommand{\pht}{\overset{\_}{t}}
\newcommand{\hypoe}[2]{\fbox{$\overset{\text{#1}}{#2}$}}
\newcommand{\hypot}[2]{\fbox{$\overset{\text{#1}}{\text{#2}}$}}
%\newcommand{\argmax}{\operatornamewithlimits{argmax}}
%\newcommand{\thickbar}[1]{\mathbf{\bar{\text{$#1$}}}}
\DeclareMathOperator{\Cat}{Cat}



% declares a document
\begin{document}



	%\title{Employee's social media use}
	%\title{Social media use by employees}
	\title{Welcome and Introduction}
	%\subtitle{for unsupervised language learning}

	\author{Miguel Rios}
	\institute[UvA]{
		%\inst{1}
		Universiteit van Amsterdam\\
	}

	\date{\today}
	
	% Title page
	{\setbeamertemplate{footline}{}
	\begin{frame}[plain]
		\titlepage
	\end{frame}
	}


	% Table of contents	
	%\frame[allowframebreaks]{
	{\setbeamertemplate{footline}{}
	\begin{frame}
		\frametitle{Content}
		\tableofcontents
	\end{frame}
	}



	% trick to start counting from the table of contents
	\setcounter{framenumber}{0}


	% SLIDES
	%\section{Word-based SMT}
\subsection{IBM models 1 and 2}
\frame{
    \frametitle{The Noisy-Channel approach}
	
	Bayes rule 
	
	$$P(E|F) = \frac{P(E)P(F|E)}{P(F)}$$
	
	Inference
	
	$$\hat{E} = \argmax_E P(E)P(F|E)$$
	
	Estimation
	
	\begin{itemize}
		\item $P(E)$ $n$-gram LM
		\item $P(F|E)$ TM
	\end{itemize}
	
}

\frame{
    \frametitle{The IBM models}

		$$P(F|E) = \sum_A P(A,F|E) $$
	\begin{center}

    	 \includegraphics[width=0.8\textwidth]{"img/wilker-align"}
	\end{center}
}



\frame{
    \frametitle{Decoding with models 1 \& 2?}
	\begin{center}
    	 \includegraphics[width=0.8\textwidth]{"img/wilker-align"}
	\end{center}
	
	~
	\pause
	how to explain insertions on the English side?
}






	\section{Introduction}

\frame[<+->]{
	\frametitle{Course details}
	
	\begin{itemize}
	\item Github course page \url{https://uva-slpl.github.io/nlp2/}
	\item Syllabus
	 \begin{itemize}
	  \item Slides
	  \item Reading material
	  \end{itemize}
	\item Projects
	\item Posts 
	\item Grading 
	 \begin{itemize}
	 \item Report in groups of 3
	 \item Project 1 \textbf{50\%} 
	 \item Project 2 \textbf{50\%} 
	 \end{itemize}
	\end{itemize}

}



\section{Natural Language Processing}
\frame[<+->]{
	\frametitle{What is NLP?}
	\begin{itemize}
	\item Goal understanding of language \\
	Not only string or keyword matching
	\item End systems 
	\begin{itemize}
	\item \cblue{Classification}: Text categorization, sentiment classification
	\item \cblue{Generation}: Question answering, Machine Translation
	\end{itemize}
	\item Computational methods to learn more about how language works (Computational Linguistics)
	\end{itemize}
	


}

\frame[<+->]{
	\frametitle{Sentiment classification}

	

}


\frame[<+->]{
	\frametitle{Natural language inference}

	

}

\frame[<+->]{
	\frametitle{Machine translation}

	

}


\frame[<+->]{
	\frametitle{Question answering}

	

}


\frame[<+->]{
	\frametitle{Graphical Models}

	

}


\frame[<+->]{
	\frametitle{Supervised learning}

	

}

\frame[<+->]{
	\frametitle{Latent variable approach}

	

}

\frame[<+->]{
	\frametitle{Latent variable Approach}
\begin{itemize}
\item Because NN models work but they may struggle with:
\item lack of training data
\item partial supervision
\item lack of inductive bias
\end{itemize}
	

}



\section{Course Topics}


\frame[<+->]{
	\frametitle{What is this course?}
	

}

\frame[<+->]{
	\frametitle{Goals}
	\begin{itemize}
	\item go through current literature
	\item define probabilistic models
	\item start combining probabilistic models and NN architectures
	
	\end{itemize}
	

}

\frame[<+->]{
	\frametitle{Next class}
	\begin{itemize}
	\item Probabilistic Graphical Models 
	\item Introduction to Word Alignment
	\end{itemize}

}







	
	%\section{Decoding}
\subsection{Complexity}
\frame{
        \frametitle{Decoding}

        Disambiguation problem
        \begin{align*}
                \hat{E} 
                &= \argmax_E P(E)P(F|E) \\
                &= \argmax_E P(E) \sum_A P(F,A|E)
        \end{align*}
        {\small \hfill NP-complete \citep{Simaan:2002:complexity}}

        \pause

        ~

        Viterbi approximation
        \begin{align*}
                \hat{E} 
                &\approx \argmax_{E, A} P(E) P(F,A|E)\\
        \end{align*}

}

\frame{
	\frametitle{Viterbi decoding}
	The alignment space (or space of \emph{derivations})
	\begin{itemize}
		\item $O(2^n)$ segmentations\\
		\item $O(n!)$ permutations\\
		\item $O(t^n)$ substitutions\\
	\end{itemize}
	\pause
	~
	
	Packed representation using finite-state transducers
	$$O(n^2 \times \alert{2^n} \times t)$$
	\hfill NP-complete (TSP) \citep{Knight:1999:tsp,Zaslavskiy+2009:tsp} 
	
	
}

\frame{
	\frametitle{Complete model}
	
	\begin{align*}
		\alert{P(E)}P(F,S|E) 
		&= \alert{\prod_{j=1}^{|E|} \psi(e_j|e_{j - n + 1}^{j - 1})} \prod_{i=1}^{|S|} \textcolor{blue}{\phi(\bar{f}_i|\bar{e}_i)} \textcolor{Green}{\delta(\text{start}_i - \text{end}_{i-1} - 1)}
	\end{align*}

	Approximations:
	\begin{itemize}
		\item distortion limit $d$: $2^n \to 2^d$
		\item maximum phrase length $m$: $n^2 \to n \times m$
	\end{itemize}
	
	~

	\begin{itemize}

		\item alignment space $O(\textcolor{Green}{2^d} \times \textcolor{blue}{n \times m}\times t )$
		\item weighted derivations $O(\textcolor{Green}{2^d} \times \textcolor{blue}{n \times m} \times t \times \alert{|\Delta|^{k-1}})$ \\
		where $P(E)$ is a $k$-gram LM components over $\Delta^*$\\
		and $|\Delta| \propto t \times n$
	\end{itemize}

	\only<2->{
	\textbf<2->{This space is too large for exact inference}
	\begin{itemize}
		\item<3> pruning: beam search
	\end{itemize}
	}
	
}

\frame{
    \frametitle{Complexity}
	 \citep{Knight:1999:tsp}
	\begin{center}
       \only<1>{
                \includegraphics[width=0.7\textwidth]{"img/knight-tsp1"}
        }
        \only<2>{
                \includegraphics[width=0.5\textwidth]{"img/knight-tsp2"}
        }
	\end{center}
}

	
	\newcounter{finalframe}
	\setcounter{finalframe}{\value{framenumber}}
	
	
	{\setbeamertemplate{footline}{}
    \begin{frame}[plain]{Questions?}
    \end{frame}
  	}
	
	\section*{References}
	
	%\input{backup}


	\frame[allowframebreaks]{ \frametitle{References}
        \bibliographystyle{plainnat}
        \bibliography{BIB}
	}

	

	% Trick to discount regular frames from the total of backup frames
	\setcounter{framenumber}{\value{finalframe}}
	
	
\end{document}
