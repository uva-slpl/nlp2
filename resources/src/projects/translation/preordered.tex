\subsection{Task 6: lattice translation}

A simple (though inefficient) way to translate a source sentence in non-monotone order would be to enumerate a alternative permutations and translate them individually.
Then, with some appropriate decision rule, we select the final translation.
Besides inefficient, some decision rules can only be defined in the combined space of translations (e.g. MBR).

Now that the permutations have been efficiently represented using transducers, translating these permutations is no different from translating a trivial linear finite-state transducer.
That is right, all it takes is a composition between the input lattice and the phrase-table transducer.
However, do notice that the parameters of the linear model have changed! This means you will have to recompile your phrase tables so that they are weighted by the correct linear model.
Table \ref{tab:task6} summarises the task. 

\begin{table}[h]\centering
\begin{tabular}{l p{12cm}}
\textsc{Task}   &  translate permutation lattices\\
\textsc{Input}  &  one permutation lattice per sentence and a recompiled wighted transducer per phrase table\\
\textsc{Output} &  a weighted set of translation derivations per sentence\\
                &  100-best paths from each transducer \\
                & \emph{best-derivation} and \emph{best-translation} decision rules\\
\textsc{Submit} & \texttt{lattice.100best.}$n$: 100-best derivations from each transducer in text format with alignments\\
                & \texttt{lattice.der} and \texttt{lattice.trans} \\  
\textsc{Report} & BLEU score for each decision rule using references in \texttt{dev.ja}\\
\end{tabular}
\caption{\label{tab:task6}Task 6 summarised}
\end{table}

