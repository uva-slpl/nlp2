\documentclass[11pt]{article}
\usepackage{geometry}                % See geometry.pdf to learn the layout options. There are lots.
\geometry{letterpaper}                   % ... or a4paper or a5paper or ... 
%\geometry{landscape}                % Activate for for rotated page geometry
%\usepackage[parfill]{parskip}    % Activate to begin paragraphs with an empty line rather than an indent
\usepackage{graphicx}
\usepackage{latexsym}
\usepackage{amsmath}
\usepackage{amssymb}
\usepackage{natbib}
\usepackage{verbatim}
\usepackage{hyperref}
\usepackage{enumerate}
\usepackage{multicol}
\usepackage{tikz}
\usetikzlibrary{bayesnet}
\hypersetup{colorlinks = true, urlcolor = blue, citecolor =red}

\DeclareMathOperator{\mS}{S}
\DeclareMathOperator{\mX}{X}
\DeclareMathOperator{\mT}{T}
\DeclareMathOperator{\mD}{D}
\DeclareMathOperator{\mI}{I}
\DeclareMathOperator{\yield}{yield}
\DeclareMathOperator{\BLEU}{BLEU}
\DeclareMathOperator*{\argmax}{arg\,max}
\DeclareMathOperator*{\argmin}{arg\,min}

\title{Project 2: Machine Translation with CRFs}

\begin{document}
\maketitle

\frame{
	\frametitle{NLP2}
	
	This course covered\pause
	\begin{itemize}
		\item Modelling techniques that power structure prediction with latent variables in NLP: mixture models, CRFs, VAE \pause
		\item Major learning techniques/algorithms: MLE and (approximate) posterior inference by VI \pause
		\item We also covered some background on logical problems \pause
			\begin{itemize} 
				\item parsing with CFGs and generalisations \pause
				\item hypergraph algorithms: inside, outside, expectation \pause
			\end{itemize}
		\item A bit of math: constrained optimisation, exponential families, semirings		
	\end{itemize}

	\pause
	
	Our latent variables:  \pause
	\begin{itemize}
		\item alignments (e.g. IBM1) \pause
		\item parameters (e.g. Bayesian IBM1)\pause
		\item trees (e.g. LV-CRF for ITGs)\pause
		\item continuous representations (e.g. VAE)
	\end{itemize}

}


\frame{
	\frametitle{What else is there?}
	
	Stuff you should learn about
	\begin{itemize}
		\item Other deep generative models: GANs		
		\item Bayesian NNs
		\item Bayesian nonparameteric models
		\item Gaussian Processes		
		\item Global optimisation
		\item Sampling: MC and MCMC		
	\end{itemize}
}

\frame{
	\frametitle{Beyond}
	

	I'll be offering a course on Bayesian NNs (stay tuned!)
	\begin{itemize}
		\item hottest topics in deep generative modelling
		\item e.g. Bayesian GANs and Bayesian VAEs
		\item a dry run might happen as a June course \\
		(email me if you are interested)
		\item lots of theory, applications somewhat toy-ish
		\item small group of students with good ML/statistics background
	\end{itemize} \pause
	
	I'm offering practical projects
	\begin{itemize}
		\item e.g. 6EC courses (1-3 students), dissertation 
		\item more applied problems
		\item findings typically lead to a publication
	\end{itemize} \pause
	
	Important
	\begin{itemize}
		\item None of it is plain supervised learning
		\item none of it is 100\% frequentist
		\item most of it about deep generative models
	\end{itemize}

}

\frame{
	\frametitle{Projects}
	
	Examples:
	\begin{enumerate}
		\item Bayesian attention: model attention weights as a random variable.
		\item Joint modelling for NMT: model a joint distribution and use monolingual data as semi-supervision.
		\item Mixture model for NMT: a mixture model to alleviate the LM bias of NMT.
		\item VAE+CRF project: discrete latent space VAE with approximation given by a CRF.
	\end{enumerate}
}

\frame{
	\frametitle{What's next?}
	
	First,
	\begin{enumerate}
		\item you finish project 3 \pause
		\item we grade project 3 \pause		
	\end{enumerate}
	
	Then, 
	\begin{itemize}
		\item I fly to a warm place and enjoy a bit of summer \pause
		\item and hopefully you too :D \pause
	\end{itemize}
	
	Finally,
	\begin{itemize}
		\item feel free to contact me in the future
	\end{itemize}

}



\begin{comment}
\section{Minimum risk training (extra)}

This section is optional and it is worth 1 extra point.

Suppose we have a loss function $l$ that can assess the quality of a hypothesised translation by comparing it to a reference (e.g. sentence-level BLEU, METEOR, BEER), then we can optimised our CRF as to minimise the expected loss (risk)
\begin{align}
	\mathcal R(w|x, y, n, w) &= \mathbb E_{P(Y|x,n,w)}[l(Y)] ~ .
\end{align}
The gradient of the risk is given by
\begin{align}
	\nabla_w \mathcal R(w|x, y, n, w) &= \mathbb E[l(Y)\phi(X,Y,D)] - \mathbb E[l(Y)] \mathbb E [\phi(X, Y, D)]
\end{align}
where all expectations are taken with respect to $P(Y,D|x, n, w)$.
On the one hand, this objective does not require parsing the reference. On the other hand, assessing the gradient requires assessing the loss for every hypothesis, which is computationally challenging for most interesting losses.\footnote{MT evaluation metrics do not typically decompose conveniently along the steps of a derivation, in fact they are mostly agnostic to the notion of a derivation.}
We can obtain a biased estimate by MC sampling, where the bias comes from the product of two estimates.
\end{comment}

\bibliographystyle{apalike}

\bibliography{../../bib}

\end{document}  
